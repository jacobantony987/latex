\documentclass{article}
\usepackage[utf8]{inputenc}

\title{Latex Certificate Course Instructions}
\author{Jacob Antony}
\date{\today}

\begin{document}
\maketitle

\section{Exercise}
Change the section `1.1 Arithmetic operations' to `1.1 Addition and Subtraction'. Also change subsection `Division is well-defined' into subsection `1.3 Division is well-defined'. And note that `1.3 Division' changes to `1.4 Division' automatically.

\section{Commands}
	A word following backslash is a \LaTeX{} command. There are 7 different commands in `day02.tex'. But, we will discuss the last three commands only.
	\begin{enumerate}
		\item[documentclass] Specifies the document class - article, letter, book, \dots. \LaTeX{} prepares a minimal document layout depending on the document class. This includes page margins and page numbering.
		\item[usepackage] This command allows us to use additional commands. There are hundreds of package available for different purposes. This includes packages for drawing images.
		\item[begin] This command is used to indicate the begin of an environment. In `day02.tex', document environment is used.
		\item[end] This command is used to indicate the end of an environment. 
		\item[section] This is the first level heading in an article document. The sections are numbered automatically in arabic numerals.
		\item[subsection] This is the second level heading in an article document. The subsections are also numbered. And the numbering of subsections will also indicate the number of its parent section.
		\item[subsection*] This a starred variant of subsection command.
	\end{enumerate}

\end{document}
