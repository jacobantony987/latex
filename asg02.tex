\documentclass{article}
% Comment the following line to NOT allow the usage of umlauts
\usepackage[utf8]{inputenc}
% Uncomment the following line to allow the usage of graphics (.png, .jpg)
%\usepackage{graphicx}

%Title Information
\title{Latex Certificate Course Instructions}
\author{Jacob Antony}
\date{\today}

\begin{document}
\maketitle

\section*{Instructions}

\begin{enumerate}
	\item Open day02.tex in Verbtex
	\begin{enumerate}
		\item Download Daily Course Material and Assignment Instructions for Day 2 (day02.tex, asg02.pdf) from Google Classroom.
		\item Open Internal storage/Android/data/verbosus.verbtex/files/Local/Latex Course/ Folder using File Manger Application
		\item Delete all files in ../Latex Course/ Folder
		\item Copy day02.tex file into Internal storage/Android/data/verbosus.verbtex/files/Local/Latex Course/  Folder
		\item Open Verbtex Application
	\end{enumerate}
	\item Exercise
	\begin{enumerate}
		\item Open day02.tex using Verbtex
		\item Change title as ``Latex Course Assignment'' ie. \textbackslash{}title\{Latex \dots \}
		\item Change author to reflect your name ie. \textbackslash{}author\{ \dots \}
		\item Save day02.tex and Generate pdf
		\item day02.tex, and Latex Course.pdf will be automatically saved in Internal storage/Android/data/verbosus.verbtex/files/Local/Latex Course/  Folder
	\end{enumerate}
	\item Upload day02.tex, and Latex Course.pdf files into Google Classroom as your response to the assignment.
\end{enumerate}

\section{\LaTeX{} Concepts}

\subsection{Logical Design}
\begin{itemize}
	\item In \LaTeX{}, document preparation is done in different stages.
		\begin{enumerate}
			\item Write your document : Text, Image, Table, \dots
			\item Design your layout  : Paper Size, Margin, Page Decorations, \dots
			\item Typeset your document : Font Style, Size, Spacing \dots
		\end{enumerate}
\end{itemize}

\subsection{Commands and Comments}
\begin{itemize}
	\item The words starting with the \textbackslash{} symbol are \LaTeX{} commands.\\ For example : \textbackslash{}documentclass, \textbackslash{}usepackage, \textbackslash{}begin, \textbackslash{}section.
	\item Everything that follows a \% symbol in a line are comments. \LaTeX{} doesn't read comments.\\ For example : \% Start the document
\end{itemize}

\subsection{\textbackslash{}begin\{document\}\dots\textbackslash{}end\{document\}}
\begin{itemize}
	\item Your document content will be in between these two lines.\\ Anything before \textbackslash{}begin\{document\} or after \textbackslash{}end\{document\} are not printed.
\end{itemize}

\subsection{Sectioning Commands}
\begin{itemize}
	\item \textbackslash{}part - not available in articles
	\item \textbackslash{}chapter - not available in articles
	\item \textbackslash{}section - $1^{st}$ Level Heading
	\item \textbackslash{}subsection - $2^{nd}$ Level Heading
	\item \textbackslash{}section - $3^{rd}$ Level Heading
	\item \textbackslash{}paragraph - $4^{th}$ Level Heading
	\item \textbackslash{}subparagraph -  $5^{th}$ Level Heading
\end{itemize}

\subsection{Starred Variants}
\begin{itemize}
	\item Every sectioning Command has a starred variant. \\ For example : \textbackslash{}section\{\dots\} has a starred variant \textbackslash{}section*\{\dots\}
	\item Starred variants skip automatic numbering.
\end{itemize}

\subsection{Errors}
\begin{itemize}
	\item \LaTeX is case-sensitive. You should use exactly the same command, without even changing uppercase/lowercase.\\ For example : \textbackslash{}Documentclass\{article\} will give an error.
	\item If \textbackslash is missing, then \LaTeX{} won't recognise the Command.
	\item If \{ \} are not properly paired, then \LaTeX{} will give an error.
	\item \LaTeX{} won't accept certain symbols \&, \textbackslash, \_, \%. There is an alternate scheme for printing these symbols.
\end{itemize}

\section*{Bonus Material}

\begin{enumerate}
	\item Try adding a few \textbf{section}, and \textbf{subsection} to this document.
	\item Try \textbf{section*}, \textbf{subsection*} and find out the difference.
	\item Try adding \% symbol at different places in your document.
	\item Visit https://www.latex-project.org/ : \LaTeX{} Official Website
	\item Visit https://en.wikipedia.org/wiki/LaTeX : wikipedia page on \LaTeX{}
\end{enumerate}

% Uncomment the following two lines if you want to have a bibliography
%\bibliographystyle{alpha}
%\bibliography{document}

\end{document}
