\documentclass{article}

\usepackage[utf8]{inputenc}
\usepackage{amsmath,amsthm}
\usepackage[inline,shortlabels]{enumitem}
\usepackage{fancyvrb}
\usepackage{graphicx}
\usepackage{hyperref}
\usepackage{subcaption}
\usepackage{tikz}
\usepackage{textcomp}
\usepackage{multicol}
%\usepackage{biblatex}

\title{Latex Certificate Course Instructions}
\author{Jacob Antony}
\date{\today}

\begin{document}

\maketitle
\section{Slide Transition}
	Advanced PDF readers support slide transitions. The following are the commands for different transition styles,
\begin{enumerate*}
	\item \textbackslash transblindshorizontal
	\item \textbackslash transblindsvertical
	\item \textbackslash transboxin
	\item \textbackslash transboxout
	\item \textbackslash transcover
	\item \textbackslash transdissolve
	\item \textbackslash transfade
	\item \textbackslash transfly
	\item \textbackslash transglitter
	\item \textbackslash transpush
	\item \textbackslash transreplace
	\item \textbackslash transsplitverticalin
	\item \textbackslash transsplitverticalout
	\item \textbackslash transsplithorizontalin
	\item \textbackslash transsplithorizontalout
	\item \textbackslash transwipe
\end{enumerate*}

	And \texttt{\textbackslash transduration} command is used inside frame environment to specify the number of seconds for the transition of particular slides generated by the frame. For example, \texttt{\textbackslash transduration$\langle2\rangle$\{3\}} sets $3$ seconds transition time for the second slide.

\section{Controlling Slides}
\subsection{Zero Slides}
	You can create frames with zero slides. For example, \texttt{\textbackslash begin\{frame\}$\langle 0 \rangle$ \dots \textbackslash end\{frame\}} is a frame with no slides for presentation. This is useful when you are using the same \LaTeX{} code for multiple purposes --- report, presentation,poster.

\end{document}

\section{Blocks}
	Beamer allows you to create blocks with minimal effort. There are three block environments --- \texttt{block}, \texttt{exampleblock} and \texttt{alertblock}.
	
	For example, \texttt{\textbackslash begin\{block\}\{Heading\}Content\textbackslash end\{block\}} will create a block with title \textit{Heading} and body \textit{Content}.
	
	The default colours for these blocks are blue, red and green for block, alertblock and exampleblock respectively.

\subsection{Mathematics}
	Beamer has a few predefined environments for mathematics. \texttt{definition}, \texttt{theorem}, \texttt{lemma}, \texttt{proof}, \texttt{corollary} and \texttt{example} are available in beamer documents. And the \texttt{amsmath} package is also loaded.


\subsection{Handout Document}
	Suppose you want to print this presentation for the audience. You can use the handout option so that the number of slides is reduced by removing redundant slides.

	The handout is generated by adding the handout optional arguemnt to the \texttt{\textbackslash documentclass} Command. That is, \texttt{\textbackslash documentclass[handout]\{beamer\}}. Thus, you can switch to handout/presentation modes instantly.

\begin{figure}[h]
\centering
\begin{subfigure}{0.45\textwidth}
\begin{Verbatim}[numbers = left]
...
\end{Verbatim}
\end{subfigure}
\begin{subfigure}{0.45\textwidth}
...
\end{subfigure} 
\caption{cap}
\label{fig:cap}
\end{figure}

consider - framezoom, links, buttons, notes, multiple screens
