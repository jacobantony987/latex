\documentclass[a4paper,12pt]{article}
% Comment the following line to NOT allow the usage of umlauts
\usepackage[utf8]{inputenc}
% Uncomment the following line to allow the usage of graphics (.png, .jpg)
%\usepackage{graphicx}

\usepackage{enumitem}
\usepackage[left=1.5in,right=1in,top=1in,bottom=0.5in]{geometry}
\usepackage{fancyhdr}
\pagestyle{fancy}

\title{Latex Certificate Course}
\author{Jacob Antony}
\date{\today}

\begin{document}

\maketitle
\tableofcontents

\section{Day 1}

\subsection{Verbtex installation}
Install Verbtex application from Google Playstore.\\
( Note : Install a pdf viewer if you don't have one.)

\subsection*{Compilation of Sample document}
Select Local Mode.\\
Create Project : \textit{Latex Course}\\
Open Project.\\
( Note : Verbtex automatically opens document.tex file. )\\
Generate pdf.

\section{Day 2}

\subsection{Document Title}
Download \textit{day02.tex} into Verbtex Project : \textit{Latex Course}\\
Open \textbf{Verbtex} Application and Open Latex Course Project.\\
( Note : Verbtex automatically opens \textit{day02.tex} file. )\\
Follow instructions from \textit{asg02.pdf}\\
Submit \textit{day02.tex} and \textit{Latex Course.pdf} files.

\section{Day 3}

\subsection{Text Formatting}
Use textbf, textit Commands for boldface, italic fonts.

\subsection{Table of Contents}
Use tableofcontents Command for auto-generating Table of Contents.\\
Always generate pdf \textbf{\textit{twice}} to update page numbers.\\
Follow instructions from \textit{asg03.pdf}

\subsection*{Combined Styles}
\addcontentsline{toc}{subsection}{Nested Commands}
This \textbf{\textit{word}} is both \textbf{boldface} and \textit{italic}.

\section{Day 4}

\subsection{Lists}
There are three list environments in LaTeX.
\subsubsection*{itemize environment}
	\begin{itemize}[label=\S]
		\item This is a bulleted list.
			\begin{enumerate}
				\item First Subitem
				\item Second Subitem
			\end{enumerate}
		\item This is second item on the list.
	\end{itemize}

\subsubsection*{enumerate environment}
	\begin{enumerate}
		\item First item
			\begin{enumerate}[label=\roman*]
				\item First Subitem
				\item Second Subitem
			\end{enumerate}
		\item Second item
		\item Third item
		\item Fourth item\footnote{Assignment Day 4}
	\end{enumerate}

\subsubsection*{description environment}
	\begin{description}
		\item[toc] table of contents
		\item[lof] list of figures
		\item[lot] list of tables
	\end{description}

\subsection{Footnotes}
This word\footnote{This is a footnote} has a footnote associated to it.

\section{Day 5}

\subsection{Page Margin}
\begin{itemize}
	\item Package geometry can be used to customize page margins.
\end{itemize}

\subsection{Header \& Footer}
\begin{itemize}
	\item Package fancyhdr can be used to customize page headers and footers.
\end{itemize}

\section{Day 6}

\subsection{Math Mode}
\begin{itemize}
	\item Inline mathematical expressions are written between two \$ symbols.
	\item There are \LaTeX{} commands that work/don't work in different settings.
\end{itemize}

\subsubsection*{Superscripts \& Subscripts}
\begin{itemize}
	\item Superscripts are written using caret \^{}
	\item Subscripts are written using underscore \_{}
\end{itemize}

\subsection*{Assignment}
	Expressions are : $3^2+4^2=5^2$, $\sin \pi = 0$, $5 \times 4 = 20$, and $a_2x^2 + a_1x+a_0$. We may write expressions with multi-character superscripts and subscripts. For example : $\log_{10} x^{12}$.

\section{Day 7}

\subsection{American Mathematical Society : AMS}
\begin{itemize}
	\item Packages : amsmath, amssymb, amsfonts, amsthm, \dots
\end{itemize}

% Uncomment the following two lines if you want to have a bibliography
%\bibliographystyle{alpha}
%\bibliography{document}

\end{document}
