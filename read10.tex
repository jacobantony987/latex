\documentclass{article}
\usepackage[utf8]{inputenc}
\usepackage{amsmath,amsthm}
\usepackage[inline,shortlabels]{enumitem}
\usepackage{fancyvrb}
\usepackage{graphicx}
\usepackage{hyperref}
\usepackage{subcaption}
\usepackage{tikz}
%\usepackage{biblatex}

\newtheorem{thm}{Theorem}[section]
\theoremstyle{definition}
\newtheorem{Def}{Definition}[section]
\theoremstyle{remark}
\newtheorem{Rmk}{Remark}[section]
\newtheorem*{Nt}{Note}

\title{Latex Certificate Course Instructions}
\author{Jacob Antony}
\date{\today}

\begin{document}

\maketitle

\section{Adding Spaces}
	\LaTeX{} has a few commands for adding horizontal and vertical spaces into your document. Also there are different mechanisms for adjusting space before and after tables, figures, paragraphs, \dots.

\subsection{hspace Command}
	The \texttt{\textbackslash hspace} command is used for adding horizontal space. For example, \texttt{Begin \textbackslash hspace\{1in\} End} gives Begin \hspace{1in} End.

\subsection{vspace Command}
	The \texttt{\textbackslash vspace} command is used for adding vertical space.
\begin{figure}[h]
\centering
\begin{subfigure}{0.45\textwidth}
\begin{Verbatim}[numbers = left]
Begin

\vspace{1in}

End
\end{Verbatim}
\end{subfigure}
\begin{subfigure}{0.45\textwidth}
Begin

\vspace{1in}

End
\end{subfigure} 
\caption{Adding vertical space}
\label{fig:vspace}
\end{figure}

\section{Phantom Space}
	\LaTeX{} has commands for creating blank spaces by adding invisible content. There are three commands --- \texttt{\textbackslash phantom}, \texttt{\textbackslash hphantom} and \texttt{\textbackslash vphantom}.
\begin{description}
\item[\texttt{\textbackslash hphantom}] creates blank space of same width
\item[\texttt{\textbackslash vphantom}] creates blank space of same height
\item[\texttt{\textbackslash phantom}] creates blank space of of same height and width.
\end{description}

\begin{figure}[h]
\centering
\begin{subfigure}{0.45\textwidth}
\begin{Verbatim}[numbers = left]
Let $x \in X$ and
$\hphantom{}_3 x_1 =
C(x,3,1)$.
\begin{equation}
\psi(ad) = \phantom{ad = }
\psi(a)\psi(d)
\end{equation}
\end{Verbatim}
\end{subfigure}
\begin{subfigure}{0.45\textwidth}
Let $x \in X$ and
$\hphantom{}_3 x_1 =
C(x,3,1)$.
\begin{equation}
\psi(ad) = \phantom{ad = }
\psi(a)\psi(d)
\end{equation}
\end{subfigure} 
\caption{Adding Phantom Text}
\label{fig:phantom}
\end{figure}

	In Figure \ref{fig:phantom} at line 2, \texttt{\textbackslash hphantom} command is used add a subscript on the left of $x$. And at line 5, \texttt{\textbackslash phantom} command is used to leave a space which is sufficient to fit $ad = $. These commands are quite useful for aligning system of equations as well as tabular data.

\section{Bibliography}
	\LaTeX{} uses \texttt{thebibliography} environment for Bibliography. The \texttt{\textbackslash bibitem} command is used to add references and to assign names. And \texttt{\textbackslash cite} command is used for references using the names assigned.

	For example, \texttt{\textbackslash cite\{ hoffman \} } gives \cite{hoffman} if you add the following lines before \texttt{\textbackslash end\{document\}}.

\begin{figure}[h]
\centering
\begin{subfigure}{0.45\textwidth}
\begin{Verbatim}[numbers = left]
\begin{document}
...
\begin{thebibliography}{1}
\bibitem{joshi}
K. D. Joshi, General Topology
\bibitem{hoffman}
Kenneth Hoffman \& Ray Kunuze,
Linear Algebra, PHI
\end{thebibliography}
...
\end{Verbatim}
\end{subfigure}
\begin{minipage}{0.45\textwidth}
\section*{Reference}
\begin{enumerate}
	\item K. D. Joshi, General Topology
	\item Kenneth Hoffman \& Ray Kunze, Linear Algebra, PHI
\end{enumerate}
\end{minipage}
\caption{thebibliography Environment}
\label{fig:bib}
\end{figure}

	The second argument, $1$ of the \texttt{\textbackslash begin\{thebibliography\}\{1\}} indicates that the document has at most $9$ references. \LaTeX{} only considers the numbers of digits of the second argument. For example, \texttt{\textbackslash thebibliography\{12\}} indicates that the document has references, the count of which is in two digits. ie, in between 10 and 99.

\begin{thebibliography}{1}
\bibitem{kdjoshi} K. D. Joshi, General Topology
\bibitem{hoffman} Kenneth Hoffman \& Ray Kunuze, Linear Algebra, PHI
\end{thebibliography}

\section{Bibliography Styles}
	The way \LaTeX{} prints the bibliography references can be changed using the \texttt{\textbackslash bibliographystyle} command. This command takes arguments \texttt{plain}, \texttt{acm}, \texttt{ieeeter}, \texttt{alpha}, \texttt{abbrv}, and \texttt{siam}.

	And \texttt{natbib} package is famous for a variety of \texttt{cite} commands and bibliography styles. The current version of this package is 8.31b and is maintained by Patrick W. Daly and Arthur Ogawa. The documentation is available at \href{https://ctan.org/pkg/natbib}{CTAN}.

\end{document}
