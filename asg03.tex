\documentclass{article}
% Comment the following line to NOT allow the usage of umlauts
\usepackage[utf8]{inputenc}
% Uncomment the following line to allow the usage of graphics (.png, .jpg)
%\usepackage{graphicx}

%Title Information
\title{Latex Certificate Course Instructions}
\author{Jacob Antony}
\date{\today}

\begin{document}
\maketitle

\section*{Instructions}

\begin{enumerate}
	\item Open day03.tex in Verbtex
	\begin{enumerate}
		\item Download Daily Course Material and Assignment Instructions for Day 3 (day03.tex, asg03.pdf) from Google Classroom.
		\item Open Internal storage/Android/data/verbosus.verbtex/files/Local/Latex Course/ Folder using File Manger Application.
		\item Delete all files in ../Latex Course/ Folder.
		\item Copy day03.tex file into Internal storage/Android/data/verbosus.verbtex/files/Local/Latex Course/  Folder.
		\item Open Verbtex Application.
	\end{enumerate}
	\item Exercises
	\begin{enumerate}
		\item Add a subsection ``Combined Styles'' before \textbackslash{}end\{document\}.
		\item Add a line ``This word is both boldface and italic.'' under the subsection ``Combined Styles''.
		\item Add boldface style to ``word'' and ``boldface''.
		\item Add italics style to ``word'' and ``italics''.
		\item Save day03.tex and Generate pdf.
	\end{enumerate}
	\item Upload day03.tex, and Latex Course.pdf files into Google Classroom as your response to the assignment.
\end{enumerate}

\section{\LaTeX{} Concepts}

\subsection{Text Formatting}
\begin{itemize}
	\item \textbackslash{}textbf Boldface Fonts\\ For example : \textbackslash{}textbf\{Bold\} produces \textbf{Bold}
	\item \textbackslash{}textit Italics Fonts\\ For example : \textbackslash{}textit\{Italics\} produces \textit{Italics}
	\item A command may be used inside another. \\  For example : \textbackslash{}textbf\{ \textbackslash{}textit\{word\}\} produces \textbf{\textit{word}}
\end{itemize}

\subsection{Line Break Command}
\begin{itemize}
	\item \textbackslash{}\textbackslash{} is used for line break.
\end{itemize}

\subsection{Table of Contents}
\begin{itemize}
	\item \textbackslash{}tableofcontents for auto-generating Table of Contents with page numbers.
	\item On first compilation, old toc file(if available in the current folder) is used for generating Table of Contents.
	\item After generating Table of Contents, page numbers are updated in day03.toc
	\item On second compilation, updated toc file is used for generating Table of Contents.
\end{itemize}

\subsection{Adding Starred Sections into Table of Contents}
\begin{itemize}
	\item Starred variants doesn't appear in Table of Contents
	\item An additional entry into the `toc' file can be made using \textbackslash{}addcontentsline\{toc\}\{\dots\}\{\dots\}
\end{itemize}

\section*{Bonus Material}

\begin{itemize}
	\item Try adding a starred section/subsection into Table of Contents without removing the star.
	\item Visit https://en.wikibooks.org/wiki/LaTeX : \LaTeX{} wikibook
\end{itemize}


% Uncomment the following two lines if you want to have a bibliography
%\bibliographystyle{alpha}
%\bibliography{document}

\end{document}
