\documentclass{article}
\usepackage[utf8]{inputenc}
\usepackage[inline,shortlabels]{enumitem}
\usepackage{fancyvrb}
\usepackage{graphicx}
\usepackage{hyperref}

\title{Latex Certificate Course Instructions}
\author{Jacob Antony}
\date{\today}

\begin{document}

\maketitle

\section{Mathematics}
\LaTeX{} was original intended to write documents with mathematical content. And it has a special mode designated for mathematics called \textbf{math mode}. \LaTeX{} changes into or returns from math mode on each occurance of \$ symbol.

\begin{figure}[h]
\centering
\begin{minipage}{0.45\textwidth}
\begin{Verbatim}[numbers = left]
...
\begin{document}
...
Let $x$ be an integer.\\
Let x be an integer.
...
\end{document}
\end{Verbatim}
\end{minipage}
\begin{minipage}{0.45\textwidth}
Let $x$ be an integer.\\
Let x be an integer.
\end{minipage} 
\caption{math mode}
\label{fig:mathmode}
\end{figure}

In figure \ref{fig:mathmode} at line 4, \LaTeX{} enters into math mode, prints $x$, then exits from math mode. But at line 5, \LaTeX{} prints x in text mode. It is better to write variable names and numbers in math mode.

\subsection{Operators}
Most of the symbols like $+,-,=,/,|$ are readily available in math mode. \LaTeX{} allows you to write complicated symbols and expressions using commands. The following are a few commonly used symbols,
\begin{table}[h]
	\centering
	\begin{tabular}{|c|c||c|c|} \hline
		Symbol & Command & Symbol & Command \\ \hline
		$\in$ & \textbackslash in & $\notin$ & \textbackslash notin \\ \hline
		$\ne$ & \textbackslash ne & $\to$ & \textbackslash to \\ \hline
		$\geq$ & \textbackslash geq & $\leq$ & \textbackslash leq \\ \hline
		$\subset$ & \textbackslash subset & $\times$ & \textbackslash times \\ \hline
		$\exists$ & \textbackslash exists & $\forall$ & \textbackslash forall \\ \hline
		$\cap$ & \textbackslash cap & $\cup$ & \textbackslash cup \\ \hline
	\end{tabular}
	\caption{Basic Symbols}
\end{table}

\begin{figure}[h]
\centering
\begin{minipage}{0.45\textwidth}
\begin{Verbatim}[numbers = left]
...
\begin{document}
...
$ 3 + 4 \times 5 = 23$\\
$ A \subset A \cup B$\\
Let $ f : X \to Y$
...
\end{document}
\end{Verbatim}
\end{minipage}
\begin{minipage}{0.45\textwidth}
$ 3 + 4 \times 5 = 23$\\
$ A \subset A \cup B$\\
Let $ f : X \to Y$
\end{minipage} 
\caption{Basic Expressions}
\label{fig:basicmath}
\end{figure}

\subsubsection{Superscript and Subscript}
In math mode, \textasciicircum and \_ are used for writing superscripts and subscripts respectively. Exponents are also written using superscripts. For example, $a^2+b_3$ is written as \$ a\textasciicircum 2 + b \_ 3 \$. Again, $1^{st}$ May is written as \$1\textasciicircum \{st\}\$ May.

\subsection{Greek Letters}

\end{document}

\begin{figure}[h]
\centering
\begin{minipage}{0.45\textwidth}
\begin{Verbatim}[numbers = left]
...
\end{Verbatim}
\end{minipage}
\begin{minipage}{0.45\textwidth}
...
\end{minipage} 
\caption{Nesting enumerate inside itemize}
\label{fig:nestedEnvironment}
\end{figure}

\subsection{Cross-referencing}
Suppose you want to refer to an image/table in the same document on a different page. \LaTeX{} does have provisions for references. You are not yet there. You should mention page number explicitly as the page numbers might change as your add/remove sentences. 
