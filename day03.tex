\documentclass{article}
\usepackage[utf8]{inputenc}

\title{Enumerating subgroups}
\author{Jacob Antony}
\date{\today}

\begin{document}

\maketitle

\tableofcontents

\section*{Prerequisite}
The reader is assumed to know the basics of abstract algebra.

\section{Introduction}
Let G be a finite, abelian group of order n. Then, G is finitely generated. 

\subsection{Enumerting distinct subgroups}
By \textit{fundamental theorem}, for each factor m of the integer n, G has atleast one subgroup of order m. Suppose m is a factor of n. Then, we want to identify \textit{distinct} subgroups of order m (if any).

\subsubsection*{m is prime power}
Suppose m is rth power of prime p. Then by fundamental theorem, there are partitions-of-r number of different groups of order m. Existence of subgroup of each kind can be \textbf{verified} by enumerating element of order p and powers of p. For example, suppose G is an abelian group of order 64. Then G has a \textit{klein-4} subgroup only if G has atleast three element of order 2. 

\subsubsection*{m is square-free}
Suppose m is a square-free integer. Then by \textit{fundamental theorem}, there is only one abelian group of order m. Thus, subgroups of order m are unique upto isomorphism.

\subsubsection*{m is not square-free}
Clearly, n is a product of power of primes. Thus G has a subgroup of order m only if G has sufficient number of lements of order p and its powers for each prime factor p of m. For example, suppose G is an abelian group of order 60. Then G has an acyclic subgroup of order 12, if G has atleast three elements of order 2.

\subsection{Enumerting isomorphic subgroups}
Now we want to enumerate subgroups that are isomorphic. This is a \textbf{simple} combinatorial problem with algebraic constraints. For example, G is an abelian group of order 72 and there are 4 elements of order 4, say a,b,c and d. Then we have two pairs (a,b) and (c,d) such that an element is always the third power of its pair. Clearly, G has an acyclic subgroups of order 8 for each pair. Thus, G has exactly two acyclic subgroups of order 8.

\end{document}
