\documentclass{article}

\usepackage[utf8]{inputenc}

\title{Latex Certificate Course Instructions}
\author{Jacob Antony}
\date{\today}

\begin{document}
\maketitle

\section{Introduction to LaTeX}
First of all, LaTeX is pronounced ``Lay-Tech''. LaTeX is a tool for creating documents, letters, presentations and posters. But, in a different style. The way we write documents in LaTeX is a bit different to the earlier methods. In LaTeX, document preparation is done in different stages.

\subsection{Start writing ?}
You might be think ``How can I start writing ? I don't know what to write ?''. The logic is simple. You wanted to write something but you don't know how to start. Start thinking of your content. Now you need a paper to write something that came to your mind. You can write that into your LaTeX document.\\
	
After some time, you will have some incomplete sentences for your document. Now you can try completing your sentences. And start rearranging and rewriting those sentences. The idea is that, in the first stage we won't worry about the way the document looks like. It might sound ``Crazy !!''. But, this is how everyone prepare to write a document. And, you can use LaTeX even before you know what to write !\\

In this stage, you can add lists, images, and tables into your document. Also you can add mathematical variables, equations and matrices in this stage. You might be thinking ``How ? I don't know yet !'' We will learn to do that in a few days. We proceed to stage 2 when we have enough scratch material to start working on.

\subsection{Document Layout}
Now you will start dividing your writing material into different sections and subsections. Again, we are not worried about the way it looks. In this stage, we are writing our document refering to the scratch material.\\

And our main concern is the document content. Sometimes, we will add new subsections or shuffle their order or add more sentences, equations, images and tables. After developing your document from the scratch, you can proceed to Stage 3.

\subsection{Decorations}
Now you can start adding some decorations to your document. You can add page numbers, adjust page margins, add line breaks and page breaks. You may also add footnotes, page headers and footers.

\subsection{Writing the Document}
Now you add more sentences to your LaTeX document. Remember that you can rewrite different parts of the  document any time you wish. LaTeX will automatically prepare the document in a consistent way.\\
	
This is the reason why people get addicted to LaTeX once they learn it. In other softwares, you will have to start with the finished document in mind. And once you shuffle the contents, all the decorations are lost. It is a tiring task to do some touch-up every time you make changes Thus, people will stop making major changes and document content will be affected.

\subsection{Typsesetting}
Typesetting is making your document ready to print. You might be wondering ``Isn't it already ready-to-print ?'' Yes, you can print it. Or you can make it look perfect. You can change fonts, add more decorations to the headings, list, tables and images. You can add cross-refereces, glossaries, index page and bibliography. You can add page decorations -- like fancy page borders if you are writing a kid's story book or coffee mug stains if you preparing a magazine. Also you can make eBooks with clickable cross-references and contents page.\\

Suppose you add another chapter or rearrage a section from one chapter into another. LaTeX will automatically update all the decorations and other details like proper section numbers, headers, footers and page numbers. Finally, you can prepare document in a great variety of output formats - not just pdf - from the same source file without much effort.

\section{Introduction to Papeeria}
Papeeria is an online solution available at `https:\textbackslash\textbackslash papeeria.com\textbackslash' which offers a much appreaciable pricing plan - `Epsilon'. Course instructor will prepare the sufficient material for this course in git. And students can easily clone the material for their practise.\\

	Students can use computer/mobile device to prepare their project document in papeeria as well.

\end{document}
