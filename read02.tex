\documentclass{article}

% We will explain this later
\usepackage[utf8]{inputenc}

% Uncomment the following line to allow the usage of graphics (.png, .jpg)
%\usepackage{graphicx}

%Title Information
\title{Latex Certificate Course Instructions}
\author{Jacob Antony}
\date{\today}

\begin{document}
\maketitle
\section{LaTeX Philosophy}
	LaTeX is a free-software.
	And is currently maintained by TUG (TeX User's Group), an online community of volunteers.
\subsection{Readable Source}
	The main file that we use to create a document is called `source' file or simply the `tex' file.
	It is because they have the `tex' file extension.
	Suppose you want to create a project report.
	Then `project.tex' is a good name for your LaTeX source file.
\subsubsection{Commands}
	``How does LaTeX print symbols that are not available on Keyboard ?''
	is an important question.
	LaTeX uses commands for this purpose.
	LaTeX commands starts with a backslash symbol.
	For example : \textbackslash LaTeX prints \LaTeX.
	Therefore, word followed by backslash is supposed to be a LaTeX command.
	If the command is already known, then LaTeX will take the necessary action instead of adding that command into your document.
	And if a command is unknown to LaTeX, it will fail to do anything further and will return an error message.


	Your document content will be in between these two lines.
	Anything before `begin-document' or after `end-document'are not printed.

\subsection{Errors}
 	LaTeX is case-sensitive.
	You should use exactly the same command, without even changing uppercase/lowercase.
	For example : Document instead of document will give an error.

	LaTeX won't accept certain symbols.


	Try adding \% symbol at different places in your document.
	Visit https://www.latex-project.org/ : LaTeX Official Website
	Visit https://en.wikipedia.org/wiki/LaTeX : wikipedia page on LaTeX

\section{Day 01 Exercise}
\begin{enumerate}
	\item Change the line ``Your text goes here'' into a comment.
	\item Add some lines of text inside document environment in a readable fashion.
\end{enumerate}


\pagebreak
\section{Appendix}
\subsection{Verbtex}
\begin{enumerate}
	\item Download Daily Course Material and Assignment Instructions for Day 1 (day01.tex, asg01.pdf) from Google Drive.
	\item Open Internal storage/Android/data/verbosus.verbtex/files/Local/Latex Course/ Folder using File Manger Application
	\item Delete all files in ../Latex Course/ Folder
	\item Copy day01.tex file into Internal storage/Android/data/verbosus.verbtex/files/Local/Latex Course/  Folder
	\item Open Verbtex Application
	\item Make necessary changes to the `day01.tex' file.
	\item Save day01.tex and Generate pdf (internet is required to generate pdf)
	\item The pdf `Latex Course.pdf' is at Internal storage/Android/data/verbosus.verbtex/files/Local/Latex Course/  Folder
	\item Submit day01.tex, and Latex Course.pdf files as response to the Google Classroom assignment.
\end{enumerate}


\end{document}
